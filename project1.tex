\documentclass[12pt]{article}


\begin{document}
  
\title{Project 1}
\author{Minh Le}
\maketitle

\section{Effective Sample Size}
\subsection{Computing $\theta$}

In this section, we examine the effectiveness of picking different $g(x)$ to compute $\theta = \int \sqrt{x^2 + y^2}\pi(x,y)dxdy$ 

\begin{itemize}
  \item For $\hat{\theta}_1$, we will draw from $\pi(x,y)$ directly.
  
  \item $\hat{\theta}_2$ will be drawn from g(x,y) a bivariate gaussian with parameters $\mu = (0,0)$ and $\sigma_0 = 1$ We assume independence between x and y.
  
  \item $\hat{\theta}_3$ will be drawn from g(x,y) a bivariate gaussian with parameters $\mu = (0,0)$ and $\sigma_0 = 1$ We assume independence between x and y.
\end{itemize}

Comparing between $\hat{\theta}_2$ and $\hat{\theta}_3$, I believe that $\hat{\theta}_3$ will converge more quickly to $\theta$. This is because both models are drawn from a distribution whose mean is not the true mean $\mu=(2,2)$ meaning the masses of $g$ and $pi$ are not distributed in the same points. It is more likely for us to draw samples close to this the mass of $\pi$ from $g_3$ because of the larger standard deviation.  

We clearly see this when plotting the estimated $theta$ for different $g$ distributions. We see that $\hat{\theta}_3$ converges to $\hat{\theta}_1$ baseline faster than $\hat{\theta}_2$ as n increases.
  
	
\subsection{effective samples size}
	ntes
\end{document}